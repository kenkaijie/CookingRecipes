% PREAMBLE
\documentclass[11pt]{article} 

%% DIMENSIONS
\usepackage{geometry} % to change the page dimensions
\geometry{a4paper} % or letterpaper (US) or a5paper or....

\usepackage[contents, index]{cuisine}

\title{Cookbook}
\author{Kenneth Kaijie Ng}
\date{}

% BODY
\begin{document}

\maketitle
 
\tableofcontents

\begin{recipe}[Meatballs]{Meatballs in Napoli Sauce}{4 Portions}{\fr34 hour}
\freeform Standard quick and dirty meatballs in a Napoli sauce.
\newstep
Preheat oven to 180\0C (160\0C fan-forced)
\ingredient[\fr34]{cup}{breadcrumbs}
\ingredient[\fr14]{cup}{unsifted plain flour}
\ingredient[\fr12]{tbsp}{dried coriander}
\ingredient[1]{tsp}{ground pepper}
\ingredient[1\fr12]{tbsp}{garlic powder}
\ingredient[1]{tsp}{paprika}
\ingredient[1]{tbsp}{onion powder}
\ingredient[1]{tsp}{salt}
To make the meatballs, in a large bowl, mix dry ingredients together. If substituting fresh herbs, add them in the next step.
\ingredient[500]{g}{beef mince (10\% fat)}
\ingredient[75]{g}{parmesan, finely greated}
In the same bowl, add the mince and parmesan. Briefly mix together.
\ingredient[2]{}{eggs, beaten}
Add beaten eggs to mixture and mix further until no ingredients sticking to the bowl.
Shape mince into balls and place into an oven tray. Place a tray of water beneath the tray of meatballs in the oven and cook for 30 minutes.
\ingredient[1]{}{onion, diced}
\ingredient[1]{tbsp}{cooking oil}
To make the sauce, sautee onions in a pan on medium heat until soft.
\ingredient[4]{tbsp}{tomato paste}
\ingredient[2]{cloves}{garlic, minced}
Add tomato paste and garlic and stir until slightly burnt.
\ingredient[400]{g}{diced or peeled canned tomatoes}
\ingredient[2]{tbsp}{italian herbs}
Add in canned tomatoes and italian herbs, then reduce heat to low and simmer for 10 minutes.
\newstep
Add meatballs to sauce and simmer for an additional 15 minutes, stirring occasionally.
\freeform\hrulefill
\end{recipe}


\end{document}
